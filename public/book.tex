\documentclass[oneside]{memoir}

\usepackage[utf8]{inputenc}

\usepackage[tracking=true]{microtype}

\usepackage{xcolor, marginnote, xparse, graphicx, caption, verse}

\usepackage[hidelinks]{hyperref}

\usepackage[hang,flushmargin]{footmisc} 

\usepackage{ucs}

\usepackage[T1]{fontenc}
\usepackage{libertine}
% \renewcommand*\familydefault{\sfdefault} 

\frenchspacing


% This rule will take exactly \baselinskip space, maintaining the grid. The
% raise value is the height above the next baseline. It will extend down
% thickness.
% \trule[thickness][raise]
\NewDocumentCommand\trule{O{0.4pt}O{0pt}}{
  \vskip0pt\vtop to0pt{
    \noindent\raisebox{#2}{\vbox{\leavevmode\hrule height#1}}}
}


%% Font Sizes

\makeatletter %only needed in preamble
\renewcommand\normalsize{\@setfontsize\normalsize{10.5pt}{18pt}}
\renewcommand\Large{\@setfontsize\Large{13pt}{18}}
\renewcommand\footnotesize{\@setfontsize\footnotesize{8pt}{11}}
\makeatother


% Bringhurst layout

\settypeblocksize{565.32431pt}{406pt}{1.25}

\makechapterstyle{Bringhurst}{%
  \chapterstyle{default}
  \setsecnumdepth{subsection}
  \renewcommand{\chaptitlefont}
               {%\fontfamily{qcr}\selectfont
               \raggedleft
                \normalfont
                \itshape
                \large}
  \renewcommand*{\printchaptername}{}
   \renewcommand*{\chapternamenum}{}
  \renewcommand*{\printchapternum}{%
  \marginnote{\chapnumfont \color{myred}}[0\baselineskip]
  }%
  \renewcommand*{\afterchapternum}{}
    \renewcommand*{\printchaptertitle}[1]{
    \chaptitlefont##1}
%  \renewcommand*{\afterchaptertitle}{\vspace{38pt}\hrule\vspace{10.5pt}} % 10.5+18-1pt for trule width

\headheight18pt
\headsep23.5pt

\beforechapskip50pt
\baselineskip18pt
\topskip0pt
\afterchapskip10.5pt

  \setsecindent{0pt}
  \setbeforesecskip{0pt}
  \setaftersecskip{18pt}
  \setsecheadstyle{
  		\clearpage
        \scshape
        \raggedright
  }

  \setaftersubsecskip{18pt}
  \setbeforesubsecskip{2.5pt}
  \setsubsecindent{0pt}
  \setsubsecheadstyle{
%  		\clearpage
        \normalfont
        \itshape
        \raggedright
  }
  


\setlength\parindent{-0em}

\setlength{\parskip}{18pt}

% \linespread{1.24}

\topmargin18pt
\oddsidemargin40pt
%\evensidemargin0pt
	
% footnotes as sidenotes

\renewcommand{\footnote}{\sidefootnote}

% \setsidefeet{ hhsepi }{ hwidthi }{ hvsepi }{ hadji }{ hfonti }{ hheighti }

\setsidefeet{10.5pt}{95.5pt}%
{\onelineskip}{0pt}%
{\normalfont\footnotesize}{\textheight}%


\makeevenhead{headings}{}{}{\thepage} 
\makeoddhead{headings}{}{}{\thepage}


}



\frenchspacing

\chapterstyle{Bringhurst}


% Poem

\renewcommand{\PoemTitleheadstart}{\clearpage}

\renewcommand{\PoemTitlefont}{\centering\scshape}

\vleftmargin0em

% cheating ;-)
%  \renewcommand{\section}{\PoemTitle*}

% Chapter numbers (?)

\addtopsmarks{headings}{}{
  \createmark{chapter}{left}{nonumber}{}{}
}
\pagestyle{headings} % activate changes



%% My Settings

\definecolor{myred}{HTML}{ab0024}


\widowpenalty=5000
 \clubpenalty=5000 
 
 \lefthyphenmin4
\righthyphenmin4

\captionsetup{labelformat=empty}



\renewcommand{\PoemTitlefont}{%
\normalfont\large\itshape}

\beforePoemTitleskip0pt
\midPoemTitleskip0pt
\afterPoemTitleskip0pt

 
\let\oldsubsection\subsection
\renewcommand\subsection{\znewpage\oldsubsection}

\let\oldsection\section
\renewcommand\section{\clearpage\gdef\znewpage{\global\let\znewpage\clearpage}\oldsection}

\global\let\znewpage\clearpage

%\renewcommand*\section\PoemTitle

% Document

\begin{document}



\chapter{Chapter 1}\label{chapter-1}

The first chapter

\section{Section}\label{section}

Paragraph\#\# Question 1.

One of the most crucial perspectives of this course is the complex
relation between theory (critical theory, theory of culture) and
theater. The approaches and discourses that are developed (by most of
the authors we have read) usually try to combine a more or less strong
critique of the European theoretical tradition (especially of the
hyperbolic status of the subject, the insistence on rationality, on
science, on representation, the repression of the body/the
physical/materiality etc.) in which `theatricality' seems to have been
repressed, with the attempt to develop a new kind of theoretical
discourse or actually: another way of writing that would account for
this theatricality. Give a critical account of this tension between
theory and theater in at least 3 texts.

\section{Three stages:}\label{three-stages}

1. Nietzsche -- beginning of the end, premonition of the consequences of
the scientific mind: ``imminence du malheur qui sommeil au sein de la
culture théorique trouble de plus en plus l'homme moderne'';

At this time Nietzsche still believes in the spirit of tragedy (Wagner);
struggle between the theoretical mind and tragic conception of the
world. Socrates, dialectic hero, optimism of science, victory of
dialectic over darkness, illusion of science -- possession of being by
knowledge, typical tendency of scientific mind to characterize, already
obvious in Sophocles, theater as peinture de caractère; victory of
appearance, Apollo = eternity of appearance. In opposition, Dionysus:
dark side, generating cause, life, beyond appearance and behind the
principles of individuation, music as idea of eternal life; in Dionysian
art ``nature speaks her own true voice without disguise''. ``Resolution
of this struggle would be Socrates practising music\ldots{}''

Modernity is age of theoretical man, signifies decline of life,
``natural cruelty of things''. ``the~Socratic culture~can
only~hold~the~scepter~of its infallibility with trembling hands''.

2. Artaud -- when life feels in a situation of radical/violent loss --
historical and metaphysical characteristic of western man (article:
`Universal Basis of Culture' from Mexico). Western culture: one that
never coincided with life. Aim of saving man by bringing him back to
life by means of theatre. Calls for change of civilization. Not so much
a discourse on crisis of western man, because a crisis is to be
resolved; but fundamental error, civilization took wrong path. Theater
must be space for rupture and transformation to be experimented and
acted. A new idea of man (total man), not a new humanism but beyond
humanism. Esp., new idea of culture, not a Pantheon but a protest
against a culture separated from life. Culture must be ``un moyen
raffiné de comprendre et exercer la vie.'' Search in non-western,
non-contemporary cultures (Mexico, Balinese theater\ldots{}) Mexique --
vieux fond métaphysique très poétique , idea of man in possession of
natural forces that ``breathe with the life of the world''.

Therefore theater must bring man back to a poetic state, through
physical means put the mind on the right path. So, theater must create
its own pure physical language made of signs and not words (utilisés
dans leur sens incantatoire); cruel because it is life (``true spectacle
of life''); must have direct and strong action on man, so it is true
action, communicates pure strong forces. Connects with reality of
imagination and dreams: the ``verso'' of the the mind. ``So obscure
facets of the spirit are revealed in real material projection''
(Sontag). Myth. Renews old conflict, in gestures. Show archetypes rather
than individual psychology. Archetypal reality is dangerous.

`Pour en finir des chef-d'oeuvres': criticism of psychology, cryptic
psychological theater, theater as mirror for audience, makes audience
into a voyeur. In the end, people yearn for mystery, so must search,
under the written poetry, for poetry itself. Anarchical power of poetry.

Go beyond closed, egotistical, personal art. ``We are not free, the sky
can still fall on our heads, and the job of theater is to teach us
that.''

3. Beckett -- after the catastrophe, when the end of the world is taken
as a matter of course (Adorno). Deconstruction of traditional aesthetic
categories attached to a metaphysical meaning, explosion of this
metaphysical meaning, explosion of the individual. Adorno: ``Beckett's
dramaturgy abandons the ontological tendency of existentialism and the
position of the absolute subject.'' Individual experience's narrowness
and contingency -- ``the prison of individuation is revealed as a
prison, a mere semblance''. Proclaims the bankruptcy of the spirit.
Thought trans formed into material of 2nd degree. Parody of forms,
parody of philosophy. Identification of subject and philosophy. Misery
of participants in Endgame is the misery of philosophy -- Adorno. Drama
itself perishes along with subjectivity. ``Post-mortem examination of
dramaturgy.'' Art can only depict solipsism, the stage becomes image of
self reflection. What remains of subject, mind and thought, in the face
of permanent catastrophe.

And then Brecht. The optimist in the story. Socratic dramatist
(Benjamin) (the optimism of marxism). Start of alternative history of
European drama. Philosophical theater, as Artaud was metaphysical
theater/consolation. Epic theater calls for a critical attitude as much
for the author as for the ``productively'' disposed spectator. Actor
adopts the attitude of a man who wanders, a man who lets himself be
amazed and create astonishment in his audience (philosophical attitude
par excellence) (Brecht). Close to lehrestück\ldots{}

Epic suggests possibility that history might have been different.
Fundamental contingency and inconsistency. The untragic hero, wise man,
thinking man, interruption: the basis of his technique; gesture is its
raw material. Gesture of showing. Story is still heart of theatrical
performance.

``The wise man is the perfect empty stage on which the contradiction of
our society is acted out\ldots{} the stage becomes a public platform.
\ldots{} Epic theater is not meant to develop action but to represent
conditions, to uncover them by means of verfremdung (alienation,
defamiliarization).

Bibliography

Friedrich Nietzsche, The Birth of Tragedy

Antonin Artaud, Le théâtre et son double

Antonin Artaud, Oeuvres Complètes, `Autour du Théâtre et son double'

Revue des lettres modernes, Modernités d'Antonin Artaud

Susan Sontag, `Approaching Artaud'

Bertolt Brecht, `A Short Organum for the Theater'

Walter Benjamin, `Understanding Brecht'

Theodor Adorno, `Trying to Understand Endgame'

%

\end{document}
